%% start of file `template.tex'.
%% Copyright 2006-2013 Xavier Danaux (xdanaux@gmail.com).
% 
% This work may be distributed and/or modified under the
% conditions of the LaTeX Project Public License version 1.3c,
% available at http://www.latex-project.org/lppl/.


\documentclass[10pt,a4paper,sans]{moderncv}        % possible options include font size ('10pt', '11pt' and '12pt'), paper size ('a4paper', 'letterpaper', 'a5paper', 'legalpaper', 'executivepaper' and 'landscape') and font family ('sans' and 'roman')

% moderncv themes
\moderncvstyle{banking}                             % style options are 'casual' (default), 'classic', 'oldstyle' and 'banking'
\moderncvcolor{blue}                               % color options 'blue' (default), 'orange', 'green', 'red', 'purple', 'grey' and 'black'
% \renewcommand{\familydefault}{\sfdefault}         % to set the default font; use '\sfdefault' for the default sans serif font, '\rmdefault' for the default roman one, or any tex font name
% \nopagenumbers{}                                  % uncomment to suppress automatic page numbering for CVs longer than one page

% character encoding
\usepackage[utf8]{inputenc}                       % if you are not using xelatex ou lualatex, replace by the encoding you are using

\input{glyphtounicode}                             % for export
\pdfgentounicode=1

\usepackage{verbatim}

% adjust the page margins
\usepackage[scale=0.94]{geometry}
\setlength{\hintscolumnwidth}{3cm}                % if you want to change the width of the column with the dates
% \setlength{\makecvtitlenamewidth}{10cm}           % for the 'classic' style, if you want to force the width allocated to your name and avoid line breaks. be careful though, the length is normally calculated to avoid any overlap with your personal info; use this at your own typographical risks...

% personal data
\name{Michael}{Stokley}
% \title{Resumé title}                               % optional, remove / comment the line if not wanted
\address{Seattle, WA}%{postcode city}{country}% optional, remove / comment the line if not wanted; the "postcode city" and "country" arguments can be omitted or provided empty
\phone[mobile]{(303) 801-8575}                   % optional, remove / comment the line if not wanted; the optional "type" of the phone can be "mobile" (default), "fixed" or "fax"
% \phone[fixed]{+2~(345)~678~901}
% \phone[fax]{+3~(456)~789~012}
\email{michael@michaelstokley.com}                               % optional, remove / comment the line if not wanted
\homepage{michaelstokley.com}                         % optional, remove / comment the line if not wanted
\social[linkedin]{mastokley}                        % optional, remove / comment the line if not wanted
% \social[twitter]{jdoe}                             % optional, remove / comment the line if not wanted
\social[github]{mastokley}                              % optional, remove / comment the line if not wanted
% \extrainfo{additional information}                 % optional, remove / comment the line if not wanted
% \photo[64pt][0.4pt]{picture}                       % optional, remove / comment the line if not wanted; '64pt' is the height the picture must be resized to, 0.4pt is the thickness of the frame around it (put it to 0pt for no frame) and 'picture' is the name of the picture file

% ----------------------------------------------------------------------------------
% content
% ----------------------------------------------------------------------------------
\begin{document}
% -----       letter       ---------------------------------------------------------
% recipient data
\recipient{Company Recruitment team}{Company, Inc.\\123 somestreet\\some city}
\date{\today{}}
\opening{Dear Hiring Manager,}
\closing{Sincerely,}
\makelettertitle

I am writing to apply for the ``Software Engineer (Remote)'' position at IXL.

My education and my professional background make me a terrific candidate for
this position. I studied both math and computer science for two years at North
Seattle College (cumulative GPA 3.65). There I gained a solid grasp of computer
science fundamentals and object-oriented design via Java, and later Python.

In addition, I recently completed a rigorous ten-week course in Python
development. Many coding bootcamps focus exclusively on popular frameworks and
developer tools at the expense of basic computer science theory. The course I
took at Code Fellows, however, placed an emphasis on data structures, basic
algorithms, and object-oriented design.

Lastly, I have spent a great deal of time independently studying from the
classic computer science textbook, \textit{The Structure and Interpretation of
  Computer Programs} (see my ongoing progress at
{\href{http://michaelstokley.com}{michaelstokley.com}}!). This has enriched, and
will continue to enrich, my understanding of functional design patterns,
function decomposition, pseudocoding, encapsulation, and recursion.

Altogether, these three avenues of education have provided me with a strong
computer science foundation and excellent programming skills in Python, an
object-oriented language.

After completing the course at Code Fellows, I helped to co-found Sternshus, a
big data startup in Seattle. I worked primarily to provide front-end solutions
using Django and Webflow. Webflow is terrific at quickly generating polished,
generic templates; integrating those templates into an existing full stack web
application, on the other hand, is time consuming. I took initiative and
designed, wrote, and tested sed and python scripts to automate the process as
much as possible. As a result, Sternshus is more agile and better equipped to
respond to rapidly changing opportunities.

Please consider the following additional qualifications. Having used Debian over
the last four years, I have a good knowledge of Unix systems. At my previous
position as an AR Specialist, I wrote and maintained a library of dozens of SQL
queries, many of which I wrote on-demand for both internal and external
customers. Finally, I have a thorough understanding of the version control
system, Git.

I encourage you to visit my personal website at
{\href{http://michaelstokley.com}{michaelstokley.com}}. There you can view some
of the projects I've been working on, such as a webscraper that uses
breadth-first traversal for fast, reliable results, or a natural language
processing script that turns tweets into poems. You may also be interested in
viewing the final project I worked on at Code Fellows; it's a full-stack Python
web application for recipe sharing and collaboration, built with Django and
PostgreSQL. The site is live at
{\href{http://reciprocity.site.}{http://reciprocity.site}}, and you can view the
source code at
{\href{http://github.com/TeamReciprocity/reciprocity}{github.com/TeamReciprocity/reciprocity}}.

Thank you for considering my application. I look forward to your response.

\makeletterclosing

\end{document}
